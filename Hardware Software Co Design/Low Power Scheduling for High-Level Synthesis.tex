\documentclass[conference]{IEEEtran}
\IEEEoverridecommandlockouts
% The preceding line is only needed to identify funding in the first footnote. If that is unneeded, please comment it out.
\usepackage{cite}
\usepackage{amsmath,amssymb,amsfonts}
\usepackage{algorithmic}
\usepackage{graphicx}
\usepackage{textcomp}
\usepackage{xcolor}
\def\BibTeX{{\rm B\kern-.05em{\sc i\kern-.025em b}\kern-.08em
    T\kern-.1667em\lower.7ex\hbox{E}\kern-.125emX}}
\begin{document}

\title{Low Power Scheduling For High-Level Synthesis\\
% {\footnotesize \textsuperscript{*}Note: Sub-titles are not captured in Xplore and
% should not be used}
% \thanks{Identify applicable funding agency here. If none, delete this.}
}

\author{\IEEEauthorblockN{1\textsuperscript{st} Rubayet Kamal}
\IEEEauthorblockA{\textit{Electronic Engineering} \\
\textit{Hochschule Hamm-Lippstadt}\\
Lippstadt, Germany\\
rubayet.kamal@stud.hshl.de}
}

\maketitle

\begin{abstract}
Low power design has become a critical factor in the technical and commercial success of modern hardware systems. High Level Synthesis (HLS) involves transforming a behavioral description into a structural RTL-level netlist through scheduling, allocation, and binding \cite{b1} \cite{b2}.The aim of this paper is to discuss integrated low power methods within the scheduling process of the HLS performed in \cite{b3}. The goal is to minimize switching activity and utilize low-power modules while meeting performance constraints, ultimately achieving a balance between size, performance, and energy efficiency. The scheduler discussed is known as the Power Scheduler \cite{b1} which identifies mutually exclusive operation paths, analyzes their activity profiles, and partitions them using a compatibility graph and clique search algorithm. Each resulting partition has a controlled activation or deactivation mechanism meaning they can be switched off when not used. 
\end{abstract}
% Traditional approaches often focus on power estimation at the RTL or gate level; however, design decisions made at the algorithmic and architectural level have a significantly greater impact on overall power consumption [x].
\begin{IEEEkeywords}
high level synthesis, scheduling, dynamic and static power, clique search algorithm.
\end{IEEEkeywords}

\section{Introduction}
This document is a model and instructions for \LaTeX.
Please observe the conference page limits. 

\section{Related Work}

\subsection{Maintaining the Integrity of the Specifications}

The IEEEtran class file is used to format your paper and style the text. All margins, 
column widths, line spaces, and text fonts are prescribed; please do not 
alter them. You may note peculiarities. For example, the head margin
measures proportionately more than is customary. This measurement 
and others are deliberate, using specifications that anticipate your paper 
as one part of the entire proceedings, and not as an independent document. 
Please do not revise any of the current designations.

\section{Fundamentals}
Before you begin to format your paper, first write and save the content as a 
separate text file. Complete all content and organizational editing before 
formatting. Please note sections \ref{AA}--\ref{SCM} below for more information on 
proofreading, spelling and grammar.

Keep your text and graphic files separate until after the text has been 
formatted and styled. Do not number text heads---{\LaTeX} will do that 
for you.

\subsection{Power Scheduling}\label{AA}
Define abbreviations and acronyms the first time they are used in the text, 
even after they have been defined in the abstract. Abbreviations such as 
IEEE, SI, MKS, CGS, ac, dc, and rms do not have to be defined. Do not use 
abbreviations in the title or heads unless they are unavoidable.

\subsection{Conclusion}
\begin{itemize}
\item Use either SI (MKS) or CGS as primary units. (SI units are encouraged.) English units may be used as secondary units (in parentheses). An exception would be the use of English units as identifiers in trade, such as ``3.5-inch disk drive''.
\item Avoid combining SI and CGS units, such as current in amperes and magnetic field in oersteds. This often leads to confusion because equations do not balance dimensionally. If you must use mixed units, clearly state the units for each quantity that you use in an equation.
\item Do not mix complete spellings and abbreviations of units: ``Wb/m\textsuperscript{2}'' or ``webers per square meter'', not ``webers/m\textsuperscript{2}''. Spell out units when they appear in text: ``. . . a few henries'', not ``. . . a few H''.
\item Use a zero before decimal points: ``0.25'', not ``.25''. Use ``cm\textsuperscript{3}'', not ``cc''.)
\end{itemize}



\section*{Acknowledgment}

The preferred spelling of the word ``acknowledgment'' in America is without 
an ``e'' after the ``g''. Avoid the stilted expression ``one of us (R. B. 
G.) thanks $\ldots$''. Instead, try ``R. B. G. thanks$\ldots$''. Put sponsor 
acknowledgments in the unnumbered footnote on the first page.

\section*{References}

Please number citations consecutively within brackets \cite{b1}. The 
sentence punctuation follows the bracket \cite{b2}. Refer simply to the reference 
number, as in \cite{b3}---do not use ``Ref. \cite{b3}'' or ``reference \cite{b3}'' except at 
the beginning of a sentence: ``Reference \cite{b3} was the first $\ldots$''

Number footnotes separately in superscripts. Place the actual footnote at 
the bottom of the column in which it was cited. Do not put footnotes in the 
abstract or reference list. Use letters for table footnotes.

Unless there are six authors or more give all authors' names; do not use 
``et al.''. Papers that have not been published, even if they have been 
submitted for publication, should be cited as ``unpublished'' \cite{b4}. Papers 
that have been accepted for publication should be cited as ``in press'' \cite{b5}. 
Capitalize only the first word in a paper title, except for proper nouns and 
element symbols.

For papers published in translation journals, please give the English 
citation first, followed by the original foreign-language citation \cite{b6}.

\begin{thebibliography}{00}
\bibitem{b1} A. Rettberg, ``Low Power Driven High-Level Synthesis
for Dedicated Architectures,'' Phd Thesis, University of Paderborn, Paderborn, North-Rhein Westphalia, December 2006.
\bibitem{b2} Daniel D. Gajski, Nikil D. Dutt, Allen C-H Wu, and Steve Y-L Lin. High-Level Synthesis. Kluwer Academic Publishers, Boston/Dordrecht/London, 1992.
\bibitem{b3} A. Rettberg and F. J. Rammig, "Integration of Energy Reduction into High-Level Synthesis by Partitioning," in IFIP International Federation for Information Processing, 2006, pp. 225-234. 
\end{thebibliography}
\vspace{12pt}
\color{red}

\end{document}
